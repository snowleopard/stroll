\documentclass[sigconf]{acmart}

\setcopyright{acmcopyright}
\copyrightyear{2021}
\acmYear{2021}

\citestyle{acmauthoryear}
\bibliographystyle{ACM-Reference-Format}

\begin{document}

\title{Stroll: a build system that doesn't require a plan}

\author{Andrey Mokhov}
\affiliation{%
  \institution{Jane Street}
  \city{London}
  \country{United Kingdom}}
% \email{andrey.mokhov@gmail.com}

\begin{abstract}
This paper presents Stroll --- a build system that doesn't require the user to
specify inputs and outputs of individual build tasks, or provide a build plan in
any other way. This makes Stroll very convenient to use, but comes at the cost
of slowing down the very first build where Stroll needs to infer the build plan
on its own. The inference works by tracking file accesses of individual build
tasks and restarting the tasks as needed to ensure the correctness of the final
build result.

The paper describes key ideas behind the implementation of Stroll, and positions
it in the landscape of existing build systems. We also quantify the cost of
inferring the build plan during the initial build.
% I call this build system Stroll because its build algorithm reminds me of
% strolling through a park where you've never been before, trying to figure out
% an optimal path to your target destination, and likely going in circles
% occasionally until you've built up a complete mental map of the park.
\end{abstract}

\begin{CCSXML}
<ccs2012>
<concept>
<concept_id>10011007</concept_id>
 <concept_desc>Software and its engineering</concept_desc>
<concept_significance>500</concept_significance>
</concept>
<concept>
<concept_id>10002950</concept_id>
 <concept_desc>Mathematics of computing</concept_desc>
<concept_significance>300</concept_significance>
</concept>
</ccs2012>
\end{CCSXML}
\ccsdesc[500]{Software and its engineering}
\ccsdesc[300]{Mathematics of computing}
\keywords{build systems, functional programming, algorithms}

\maketitle

\section{Introduction}
Most build systems require the user to specify both \emph{build tasks} as well
as \emph{dependencies} between the tasks. The latter essentially provides the
build system with a ``plan'': by building the tasks in an order that respects
the dependencies, the build system can guarantee that no task is executed more
than once~\cite{mokhov2020build}.

Describing dependencies between tasks can be problematic for two reasons:

\begin{itemize}
  \item Providing an accurate description of dependencies, and then keeping the
  description up to date as build tasks evolve, is difficult and is often a
  source of frustration and subtle correctness and performance
  bugs~\cite{spall2020build}.

  \item To describe dependencies, one needs a suitable domain-specific language,
  and indeed (almost) every build system comes with its own:
  Make~\cite{feldman1979make} uses makefiles, Bazel~\cite{bazel} uses a
  Python-inspired language called Starlark~\cite{starlark},
  Shake~\cite{mitchell2012shake} uses a Haskell EDSL, Dune~\cite{dune} uses a
  combination of OCaml and a higher-level S-expression based configuration
  language, etc. While learning a new task description language is not a big
  deal, migrating an existing build system to a new language may take
  years\footnote{The author was involved in two such migrations and both are
  still under way after investing multiple man-years; the first one -- migrating
  GHC's build system from Make to Shake~\cite{hadrian} -- started in 2014!}.
\end{itemize}

\noindent
Stroll takes a radically different approach. Given a collection of build tasks,
it treats them as black boxes and discovers dependencies between them by
executing the tasks and tracking their file accesses. This approach is not
optimal in the sense that a task may fail because one of its dependencies has
not yet been built. A task may therefore need to be restarted multiple times
until all of its dependencies have been discovered and brought up to date. In
the end, Stroll will learn the complete and accurate dependency graph and will
store it to speed up future builds.

At the first glance, this approach may seem hopelessly slow. As we show in this
paper, in the worst case, the number of restarts that Stroll's algorithm needs
to perform is linear with respect to the size of the dependency graph.
Furthermore, the restarts are not on the critical path, which means one can
recover performance by giving Stroll a sufficient number of parallel workers. We
also show that a simpler version of the problem requires at most 2x work
compared to the optimum.

Stroll was inspired by Fabricate~\cite{fabricate} that also tracks file accesses
to automatically compute accurate dependencies. Fabricate itself was preceded by
Memoize~\cite{memoize} and succeeded by Rattle~\cite{spall2020build}. Unlike
Stroll, all these build systems require the user to provide a build plan by
listing the tasks in a topological order. Stroll takes the idea of file access
tracking to the limit and doesn't require any plan, thus occupying a unique
point in the design space of build systems.

\subsection*{Note for IFL 2021 reviewers}

This is a last-minute submission that only describes two key algorithms used in
Stroll, to make it possible for the reviewers to evaluate the performance
claims. If this submission is accepted for a presentation at IFL 2021, the
presentation will include a high-level Haskell model of Stroll, building on the
modelling framework from~\cite{mokhov2020build}. The final version of the paper
will include: (i) a Stroll model, elaborated further to highlight interesting
aspects of Stroll's implementation (Stroll is written in Haskell); (ii)
illustrated examples taken from the original blog post about
Stroll~\cite{stroll}, clarifying how Stroll discovers the full dependency graph
by executing and restarting ``black box'' tasks; and (iii) proofs of correctness
and efficiency of Stroll's build algorithms.

\section{Stroll's main algorithm}

Stroll uses a (slightly optimised and therefore less obviously correct) version
of the following algorithm.

\begin{itemize}
  \item For each previously executed task $x$, store a \emph{trace} $T_x$ and a
  \emph{status} $S_x$. The trace lists all inputs and outputs of the task
  recorded during its last execution. The status is either \emph{success} or
  \emph{failure}, depending on the last exit code.

  \item A task $x$ is \emph{complete} if $S_x=\textit{success}$ and all inputs
  in $T_x$ are complete. If one of the inputs in $T_x$ is not complete, the task
  is \emph{blocked} (on the corresponding input).

  \item Execute tasks in parallel \emph{rounds}. In each round, execute all
  tasks that are not complete and not blocked. (Note that in the first round all
  tasks are executed in parallel.)

  \item Terminate as soon as the build target requested by the user has been
  produced, and the task that produced it is complete.

  \item Report an error if the current execution round is empty: the target
  cannot be built (e.g., due to a cyclic dependency).
\end{itemize}

\vspace{2mm}
\noindent
\textbf{Claim:} For a dependency graph with $n$ tasks and $m$ dependency edges,
there will be at most $n$ rounds and at most $m$ task restarts.

\vspace{2mm}
\noindent
\textbf{Proof sketch:} Our first observation is that if a task gets executed in
a round $R$ and does not become complete, then it will need to be restarted at
some later round $R' > R$, when the dependency that currently blocks it becomes
complete. A restart therefore requires at least one dependency edge switching
from blocked to complete, and since there are $m$ edges overall, there will be
at most $m$ restarts. The second observation is that for an edge to switch from
blocked to complete, at least one task must become complete. Therefore, there
cannot be more than $n$ rounds, since every round is non-empty, which requires
at least one task to become complete. In fact, this bound can be strengthened:
there will be as many rounds as the number of tasks in the longest dependency
chain.

\section{Building tasks with known outputs}

While Stroll doesn't require any information about tasks, providing \emph{some}
information may help to significantly speed up the first build. For example, if
there is a mapping from outputs to the corresponding tasks, then it is possible
to bound the number of restarts by just~$n$ for a build graph with $n$ tasks
and $m$ dependencies.

Consider the following algorithm.

\begin{itemize}
  \item Execute the task $x$ corresponding to a requested build target.

  \item If the task doesn't get blocked, terminate the build. If the task is
  complete, then the target has been successfully built; otherwise, it can't be
  built ($x$ must have failed with an error).

  \item If $x$ gets blocked on another task $y$, switch to building $y$. If that
  gets blocked too, keep building dependencies in the depth-first manner. If
  there are no cycles, a ``leaf'' task with no dependencies will eventually be
  reached, and the previously blocked task will be restarted. Continue until all
  tasks blocking the ``root'' $x$ have been built.
\end{itemize}

\vspace{2mm}
\noindent
\textbf{Claim:} The algorithm does at most $n$ task restarts.

\vspace{2mm}
\noindent
\textbf{Proof sketch:} We can count the number of restarts by counting the
number of blocking edges discovered by the above algorithm, since every time a
task is blocked, we recurse to build the blocking dependency and then restart
the task. Here is a simple observation: each task can be a blocking dependency
at most once. This holds because we build tasks in the depth-first order: when
we ``enter'' a task, we never ``leave'' it until its whole subtree is complete,
and so only the very first incoming edge will be blocking. Since there are $n$
tasks and each has at most one incoming blocking edge, there will at most $n$
blocking edges, and hence at most $n$ restarts.

\bibliography{refs}
\end{document}
